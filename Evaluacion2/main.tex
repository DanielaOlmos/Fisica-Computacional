% Don't modify this section unless you know what you're doing!
\documentclass[letterpaper,12pt]{article}
\usepackage{float}
\usepackage[spanish]{babel}
\selectlanguage{spanish}
\usepackage[utf8]{inputenc}
\usepackage{tabularx} % extra features for tabular environment
\usepackage{amsmath}  % improve math presentation
\usepackage{graphicx, wrapfig, subcaption, setspace, booktabs}
\usepackage{graphicx} % takes care of graphic including machinery

\usepackage[margin=1in,letterpaper]{geometry} % decreases margins
\usepackage{cite} % takes care of citations
\usepackage[final]{hyperref} % adds hyper links inside the generated pdf file
\usepackage{amsmath}
\usepackage{amssymb}
\usepackage{enumerate}
\usepackage{url}
\hypersetup{
	colorlinks=true,       % false: boxed links; true: colored links
	linkcolor=blue,        % color of internal links
	citecolor=blue,        % color of links to bibliography
	filecolor=magenta,     % color of file links
	urlcolor=blue         
}
%++++++++++++++++++++++++++++++++++++++++


\begin{document}

\title{Reporte de la Evaluación 2}
\author{Daniela Olmos Velderrain\\Grupo 3}
\date{5 de mayo de 2019}

\maketitle

\section{Introducción}
En esta actividad se utilizaron datos meteorológicos para estimar la evapotranspiración en un viñedo ubicado en el kilómetro 41 de la carretera de Hermosillo a Bahía Kino.\\\\
La evapotranspiración es un parámetro muy importante para determinar el manejo de agua en la agricultura, por lo cual se han desarrollado una variedad de ecuaciones para su cálculo bajo distintas condiciones climáticas. \\\\
La presente actividad se dividirá en tres partes, y en cada una se desarrollarán procedimientos encaminados a calcular la evapotranspiración del viñedo en cuestión. \\\\
En la primera parte se trabajará sobre una serie de datos metorológicos del año 2018, donde se reproducirá la tabla presente en el artículo de Koffi Djaman titulado Evaluation of the Penman-Monteith and other 34 reference evapotranspiration equations under limited data in a semiarid dry climate.
\\\\
En la segunda parte se utilizarán los datos de dicha tabla para calcular la evapotranspiración mediante tres ecuaciones distintas.
\\\\
Finalmente, en la tercera parte se realizará una gráfica para observar el balance de energía para un mes en el viñedo. 
\\\\
En la estructura de este reporte se mostrará una sección de desarrollo, que incluye el marco teórico y la metodología, donde se explica de manera breve el procedimiento seguido. En la sección de resultados se presentan todas las tablas y gráficas generadas durante la actividad, y en la conclusión se presentan comentarios generales sobre los productos obtenidos y la realización de la actividad.
    
\section{Desarrollo}
\subsection{Marco teórico}
La evapotranspiración se define como la pérdida de humedad de una superficie por evaporación directa junto con la pérdida de agua por transpiración de la vegetación.\\\\
En términos aplicados, quizás una de las más conocidas referencias al fenómeno venga de la climatología y de la consideración y utilidad de la evapotranspiración como un indicador de aridez de las distintas zonas.\\\\
Sin embargo, donde la evapotranspiración ha ganado un lugar realmente importante es en la evaluación de los volúmenes de agua involucrados, aspecto indispensable en las tareas de planificación y gestión de los recursos hídricos, en ciertos estudios medioambientales y en la cuantificación de las demandas hídricas de la vegetación, especialmente de los cultivos. 


\subsection{Metodología} 
La metodología de esta evaluación es similar a la de la Actividad 8, ya que se trabajó con dos conjuntos de datos, pero esta vez por separado. Para explicar el procedimiento, se dividirá la actividad en 3 partes.

\subsubsection{Parte 1}
Se leyó el archivo \emph{meteo-vid-2018.csv} con los datos meteorológicos y se guardó la información en un dataframe.\\\\
Después, se creó una variable de fecha a partir de dos columnas que contenían información sobre el día y la hora.\\\\
Para esta parte de la actividad se solicitó calcular el promedio mensual de siguientes datos:

\begin{enumerate}
    \item Latitud (º N) 
    \item Longitud (º O)
    \item Elevación (m)
    \item Velocidad del Viento (m/s)
    \item Temperatura máxima ($^{\circ}C$)
    \item Temperatura mínima ($^{\circ}C$)
    \item Humedad relativa máxima (\%)
    \item Humedad relativa mínima (\%)
    \item Humedad relativa promedio (\%)
    \item Radiación neta ($MJ/m^{2}/día$)
\end{enumerate}

Para calcular los parámetros solicitados, se redujo el dataframe a las columnas que contenían esta información. Posteriormente, mediante las funciones \emph{groupby} y \emph{transform} se calcularon los valores máximos y mínimos solicitados para cada día de temperatura y humedad relativa.\\\\ 

Después de agregar las columnas con esta información al dataframe, empleamos las funciones antes mencionadas para calcular un promedio mensual de todos los datos.\\\\

Con los datos obtenidos, se realizaron gráficas para las temperatura, humedad relativa y radiación solar durante todo el año.

\subsubsection{Parte 2}
En la segunda parte se solicitó estimar la Evapotranspiración ($ET_{0}$) mensual promedio, basándonos en los resultados obtenidos en la primera parte.
\\\\
Para realizar este cálculo se emplearon las ecuaciones de 3 autores diferentes:

\begin{enumerate}
    \item Ecuación 7, Jansen & Haise (1963):
    
    \[ET_{0} = (0.0252Tmean + 0.078)Rs\]
     
     - Tmean es la temperatura promedio.\\\\
     - Rs es la radiación solar.

    \item Ecuación 31, Valiantzas 1 (2012):

    \[ET_{0} = 0.0393 Rs (Tmean + 9.5)^{0.5} - 0.19(Rs^{0.6})(\varphi^{0.15}) +\] \[0.0061(Tmean + 20)(1.11Tmean - Tmin - 2)^{0.7}\]
     
     - $\varphi$ es la latitud en radianes.\\\\

    \item Ecuación 32, Valiantzas 4 (2013):

    \[ET_{0} = 0.051(1 - \alpha)Rs(Tmean + 9.5)^{0.5} - 2.4(Rs/Ra)^2 +\] 
    \[0.048(Tmean + 20)(1 - RH/100)(0.5 + 0.536u2) + 0.00012z\]

    - $\alpha$ es el albedo.\\\\
    - $u2$ es la velocidad del viento a 2 m de altura.\\\\
    - $z$ es la altura sobre el nivel mar.\\\\
    - $Ra$ es la radiación solar en la parte alta de la atmósfera. Ésta se calcula mediante la siguiente ecuación:
    
        \[Ra = 24(60)/\pi (Gsc)(dr)(\omega sin\varphi sin\delta + cos\varphi cos\delta sin\omega)\]

    - $Gsc$ es la constante solar ($Gsc=0.0820$ $MJm^{-2}min^{-1}$)\\\\
    - $dr$ es la distancia relativa entre la Tierra y el Sol.\\\\
    - $\omega$ es el ángulo de la hora donde se oculta el sol.\\\\
    - $\varphi$ es el latitud ($rad$).\\\\
    - $\delta$ es el declinación solar ($rad$).\\\\
\end{enumerate}
Los cálculos fueron realizados para cada mes, y fueron registrados en un dataframe para comparar los resultados de las tres ecucaciones empleadas.


\subsubsection{Parte 3}
Para esta parte se trabajó sobre el archivo \emph{flujos-vid-2018.csv}. En este archivo fue necesario modificar la fecha, ya que los días se encontraban escritos de acuerdo al calendario juliano (al igual que el archivo de la actividad 8).\\\\\
Una vez arreglada la fecha, se pudo trabajar sobre los datos de un mes (febrero en este caso) para visualizar una gráfica de Balance de energía, el cual corresponde a la siguiente ecuación:

\[Rn - G - \lambda ET - H = 0\]

Donde $Rn-G$ es la radiación neta, $\lambda ET$ es el calor latente y $H$ es el calor sensible.\\\\

Para calcular los parámetros se agruparon los datos por hora para el mes de febrero. Se realizaron dos casos, el primero para un solo día y el segundo para el mes entero.



\subsection{Resultados}

\subsubsection{Parte 1}
La tabla de promedios mensuales para el viñedo fueron los siguientes:

\begin{figure}[H]
\centering
\includegraphics[width=1\textwidth]{tabla_param.png}
\caption{\label{fig:graf1}: Promedios mensuales para un viñedo ubicado en el kilómetro 41 de la carretera de Hermosillo a Bahía Kino (Latitud 28\º 55.117' N, Longitud 111\º 18.638' W, altitud 101m).}
\end{figure}


A partir de esta tabla se realizaron las gráficas para temperatura, humedad relativa y radiación solar mensuales mostradas a continuación.

\begin{figure}[H]
\centering
\includegraphics[width=0.7\textwidth]{T_2019.png}
\caption{\label{fig:graf2}: Promedio mensual de temperaturas en 2018.'}
\end{figure}

\begin{figure}[H]
\centering
\includegraphics[width=0.7\textwidth]{HR_2019.png}
\caption{\label{fig:graf3}: Promedio mensual de humedad relativa en 2018.}
\end{figure}

\begin{figure}[H]
\centering
\includegraphics[width=0.7\textwidth]{Rs_2019.png}
\caption{\label{fig:graf1}: Promedio mensual de radiación solar en 2018. }
\end{figure}


\subsubsection{Parte 2}
En esta parte se obtuvo una tabla de comparación para los valores de $ET_{0}$ calculados con tres ecuaciones planteadas por distintos autores.

\begin{figure}[H]
\centering
\includegraphics[width=0.5\textwidth]{eto.png}
\caption{\label{fig:graf1}: Estimación de la evapotranspiración de acuerdo a las ecuaciones de Jansen & Haise (Ec.7) y, Valiantzas (Ec. 31 y 34). }
\end{figure}

\subsubsection{Parte 3}
Aquí se obtuvieron gráficas que muestran el balance de energía (energía que entra y sale de la superficie) para un día y para un mes.

\begin{figure}[H]
\centering
\includegraphics[width=0.5\textwidth]{balance_enero.png}%
\includegraphics[width=0.5\textwidth]{balance_febrero1.png}
\caption{\label{fig:graf1}: Balance de energía promedio en un día y en un mes.}
\end{figure}
\newpage


\section{Conclusiones}

En la primera parte se observa que tanto las temperaturas como los índices de humedad parcen ser congruentes con el lugar y los meses del año en que fueron registrados, ya que las temperaturas y radiación solar son mayores en los meses de verano, y la humedad incrementa en la época de lluvias característica de la región.
\\\\
En la segunda parte los cálculos de la evapotranspiración parecen variar de acuerdo a la ecuación utilizada, especialmente en la ecuación 7, donde la estimación de dicho parámetro es de mayor magnitud que las otras dos.
\\\\
En la parte tres se obtuvieron dos gráficas, la primera para el balance de energía durante un día en febrero y la segunda para todo el mes de febrero. El comportamiento de las gráficas es similar, ya que los parámetros graficados fueron promediados por su respectiva hora. Sin embargo, la gráfica para el mes entero tiene un comportamiento claramente más uniforme. 
\\\\
De manera general, las parte que ocasionó más complicaciones en esta actividad fue la estimación de $ET_{0}$ mediante las ecuaciones, ya que algunas presentaban muchos parámetros en sus cálculos, lo cual puede haber sido una fuente de error. Apesar de ello, fue interesante para aprender sobre las distintas variables que se necesitan tomar en cuenta en actividades tan importantes como el manejo del agua en la agricultura. 






\section*{Bibliografía}
\begin{itemize}

\item \\Evapotranspiración. Recuperado el 5 de mayo de 2019 desde \\https://es.wikipedia.org/wiki/Evapotranspiraci\%C3\%B3n
\\

\item \\ Djaman,K. (2018) \emph{Evaluation of the Penman-Monteith and other 34 reference evapotranspiration equations under limited data in a semiarid dry climate}. Theoretical and Applied Climatology.

\end{itemize}

\end{document}
